\documentclass{article}

\usepackage[utf8]{inputenc}
\usepackage[T1]{fontenc}
\usepackage[polish]{babel}
\usepackage[a4paper, margin=1in]{geometry}
\usepackage{multicol}
\usepackage{amsmath}
\usepackage{gensymb}
\usepackage{graphicx}
\usepackage{multirow}
\usepackage{array}
\usepackage{caption}
\usepackage{float}
\usepackage{hyperref}
\usepackage{karnaugh-map}
\usepackage{tikz}
\usepackage[linesnumbered,ruled,vlined]{algorithm2e}

\begin{document}

\begin{multicols}{3}
    \begin{figure}[H]
        \includegraphics[scale=0.4]{jpg/DOJEBANE_LOGO_PWR.png}
        \label{fig:enter-label}
    \end{figure}
    
    \begin{figure}[H]
    \end{figure}
    
    \begin{figure}[H]
        \includegraphics[scale=0.4]{jpg/w4n.png}
        \centering
        \textbf{W4N}
        \label{fig:WYDZIAŁ INFORMATYKI I TELEKOMUNIKACJI}
    \end{figure}
\end{multicols}

\vspace{75pt}

\begin{center}
    \textbf{\large Porównanie efektywności operacji arytmetycznych wektorowych SIMD i zwykłych SISD} 
    \vspace{2pt}
    \hrule
    \vspace{4pt}
    \textbf{\large Organizacja i Architektura Komputerów} 
\end{center}

\vspace{75pt}

\begin{center}
    \textbf{Autor: } \\
    Filip Kwiek 280947
\end{center}

\begin{center}
    \textbf{Termin zajęć: } \\
    Środa np. 9:15
\end{center}

\begin{center}
    \textbf{Prowadzący: } \\
    Mgr. inż. Tomasz Serafin
\end{center}

\newpage
\tableofcontents
\newpage

\section{Wprowadzenie}

Celem eksperymentu było porównanie efektywności czasowej operacji arytmetycznych wektorowych i niewektorowych.

\subsection{Rys historyczny}

\subsection{Operacje SISD i SIMD}

\section{Użyty sprzęt}

\begin{itemize}
    \item Procesor: AMD Ryzen 5 3550h
    \item System Operacyjny: Arch linux
    \item Kernel: 6.4.14-zen1-2-zen
    \item Ram: 32Gb DDR4 2666MHZ 
\end{itemize}

\section{Przebieg pracy nad programem}

Na labolatoriach wprowadzających stworzyłem funkcje odpowiadające za operacje oparte na wektorach, napisane za pomocą wstawek assemblerowych w moich programach pisanych w C.
Następnie w domu przez resztę czasu stworzyłem funkcje odpowiadające za operacje SISD, obsługę obu typów operacji, a następnie tworzenie pliku z danymi, które powstały w wyniku badania.
Ponadto do obróbki danych powstał skrypt w pythonie, który odpowiada za generację wykresów na podstawie danych z pliku.

\section{Uruchamianie programu}

Na potrzeby programu powstał makefile, który odpowiada za kompilację wszystkich plików i został stworzony na podstawie 
zasobów dostępnych w internecie:

\begin{verbatim}
    # Compiler
    CXX = g++
    # Compiler flags
    CXXFLAGS = -Wall -g
    # Include paths (add -I flags for header directories if needed)
    INCLUDES = -I.
    # Target executable
    TARGET = main
    # For deleting the target
    TARGET_DEL = main.out
    
    # Find all source files recursively
    SRCS := $(shell find . -name "*.cpp")
    # Generate object file paths (preserving directory structure)
    OBJS := $(SRCS:.cpp=.o)
    # Find all header files recursively (for dependency generation)
    HEADERS := $(shell find . -name "*.h")
    
    # Default rule to build and run the executable
    all: $(TARGET) run
    
    # Rule to link object files into the target executable
    $(TARGET): $(OBJS)
        $(CXX) $(CXXFLAGS) -o $(TARGET) $(OBJS)
    
    # Rule to compile .cpp files into .o files with automatic header dependencies
    %.o: %.cpp
        $(CXX) $(CXXFLAGS) $(INCLUDES) -MMD -MP -c $< -o $@
    
    # Include dependency files
    -include $(OBJS:.o=.d)
    
    # Rule to run the executable
    run: $(TARGET)
        ./$(TARGET)
    
    # Clean rule to remove generated files
    clean:
        del $(TARGET_DEL) $(OBJS) $(OBJS:.o=.d)
\end{verbatim}

W folderze źródłowym, tym samym gdzie znajduje się makefile należy uruchomić polecenie: $make$.
Po wpisaniu polecenia, program się uruchomi i wpisze dane do pliku z wynikami pomiarów czasu operacji.

\section{Napotkane problemy}
Jedyne problemy na jakie się natknąłem podczas pisania programu, to mój własny brak wiedzy, tj. nieznajomość, w jaki sposób należy się odnosić 
to tablic zapisanych w pamięci w wstawkach assemblerowych, ale po przeczytaniu dokumentacji, przestało to stanowić problem.

\section{Przebieg eksperymentu}

W obu wypadkach działałem na liczbach zmiennoprzecinkowych, które był generowane pseudolosowo
za pomocą następujących funkcji z biblioteki $random$:

\begin{verbatim}
    std::random_device rd;
    std::mt19937 gen(rd());
    std::uniform_real_distribution<float> distribution(0, 1000);
\end{verbatim}

Następnie mierzyłem czas za pomocą funkcji $std::chrono::high\_resolution\_clock::now()$ z biblioteki $chrono$.

\section{Operacje SIMD}

\subsection{Dodawanie:}
\begin{verbatim}
    double gen_add_vectors(int n)
{
    // data generation
    std::random_device rd;
    std::mt19937 gen(rd());

    std::uniform_real_distribution<float> distribution(0, 1000);
    auto startTime = std::chrono::high_resolution_clock::now();

    for (int i = 0; i < n / 4; i++)
    {
        float vec1[4];
        float vec2[4];

        for (int j = 0; j < 4; j++)
        {
            vec1[j] = static_cast<float>(distribution(gen));
            vec2[j] = static_cast<float>(distribution(gen));
        }

        __asm__ volatile(
            "movaps (%0), %%xmm0\n"
            "movaps (%1), %%xmm1\n"
            "addps %%xmm1, %%xmm0\n" 
            :                       
            : "r"(vec1), "r" (vec2)
            : "xmm0", "xmm1" 
        );
    }

    auto endTime = std::chrono::high_resolution_clock::now();
    auto duration = std::chrono::duration<double, std::milli>(endTime - startTime).count();

    return duration;
}
\end{verbatim}

W powyższym kodzie najpier następuje generacja losowych liczb zmiennoprzecinkowych, które następnie wpisywane są do dwóch tablic 
czterolelementowych. Następnie dodaję te dwa wektory do siebie za pomocą wstawki assemblerowej.

Pozostałe operacje są robione w taki sam sposób różniąc się jedynie wstawkami assemblerowymi.

\subsection{Odejmowanie:}

\begin{verbatim}
    __asm__ volatile(
        "movaps (%0), %%xmm0\n"
        "movaps (%1), %%xmm1\n"
        "subps %%xmm1, %%xmm0\n" 
        :                       
        : "r"(vec1), "r" (vec2)
        : "xmm0", "xmm1" 
    );
\end{verbatim}

\subsection{Mnożenie:}

\begin{verbatim}
    __asm__ volatile(
        "movaps (%0), %%xmm0\n"
        "movaps (%1), %%xmm1\n"
        "mulps %%xmm1, %%xmm0\n" 
        :                       
        : "r"(vec1), "r" (vec2)
        : "xmm0", "xmm1" 
    );
\end{verbatim}

\subsection{Dzielenie:}

\begin{verbatim}
    __asm__ volatile(
        "movaps (%0), %%xmm0\n"
        "movaps (%1), %%xmm1\n"
        "divps %%xmm1, %%xmm0\n" 
        :                       
        : "r"(vec1), "r" (vec2)
        : "xmm0", "xmm1" 
    );
\end{verbatim}

\subsection{Wykresy}

W wyniku eksperymentu powstały następujące wykresy:



\section{Operacje SISD}

\section{Wnioski}

\end{document}